\documentclass[11pt]{scrartcl}

\usepackage[top=2cm]{geometry}
%\pagestyle{empty}

\title{
  \textbf{\large Database Tuning -- Assignment 5}\\
  Join Tuning
}


\author{
A2\\
\large Baumgartner Dominik, 0920177 \\
\large Dafir Thomas Samy, 1331483 \\
\large Sch\"orgnhofer Kevin, 1421082
}

\begin{document}

\maketitle
\section{Setup}
All queries were sent to the database-server (biber) using $psql$ on the computers of the R\"UR.

\section{Join Strategies Proposed by System}

\paragraph{Response times}

\begin{flushleft}
\begin{tabular}{l|l|l}
  Indexes & Join Strategy Q1 & Join Strategy Q2\\
  \hline
  no index & Hash Join & Hash Join  \\
  unique non-clustering on {\tt Publ.pubID} & Hash Join  & Nested Loop Join \\
  clustering on {\tt Publ.pubID} and {\tt Auth.pubID} & Merge Join & Nested Loop Join \\
\end{tabular}
\end{flushleft}

\paragraph{Discussion}
Discuss here your observations. Is the choice of the strategy
expected? How does the system come to this choice?\\
no index:\\
For this scenario the proposed join strategies are the expected ones. A hash join is in this case the best choice. For a merge join the tables should be sorted and the nested loop join should be used when there is a small table.\\
\\
unique non-clustering on {\tt Publ.pubID}:\\
For the first query a hash join is the best choice, because of the big, unsorted table of both. For this query the index is of no use.\\
For the second query a Nested Loop Join is used because first the name from the author must be searched which reduces the table size from $3*10^6$ to $\sim$200. Thereforce the nested loop join is the better choice.\\
\\
clustering on {\tt Publ.pubID} and {\tt Auth.pubID}:\\
For the first query a Merge Join is proposed as expected. The reason therefore is that both tables are sorted to this attribute which is ideal for the Merge Join.\\
For the second query we expected a Merge Join, but the system used a Nested Loop Join. The Reason therefore is again the table size. First the name is searched in Auth which reduces the size and also "destroys" the ordering. This means that the system would have to sort the values again for the Merge Join although the values are still sorted. And therefore a Nested Loop Join is the best choice.


\section{Nested Loop Join}

\paragraph{Response times}

\begin{flushleft}
\begin{tabular}{l|r|r}
  Indexes & Response time Q1 [ms] & Response time Q2 [ms] \\
  \hline
  index on {\tt Publ.pubID} & 98635 & 504  \\
  index on {\tt Auth.pubID} & 46672 & 193800 \\
  index on {\tt Publ.pubID} and {\tt Auth.pubID} & 56395 & 511 \\
\end{tabular}
\end{flushleft}

\paragraph{Query plans}\mbox{}\\ 

\noindent Index on {\tt Publ.pubID} (Q1/Q2):
{\small
\begin{verbatim}
Q1:
 Nested Loop  (cost=0.43..1589599.47 rows=3095201 width=82) 
 (actual time=0.088..97373.109 rows=3095201 loops=1)
   ->  Seq Scan on auth  (cost=0.00..57761.01 rows=3095201 width=38) 
       (actual time=0.012..1694.669 rows=3095201 loops=1)
   ->  Index Scan using pubidp on publ_i  (cost=0.43..0.48 rows=1 width=89) 
       (actual time=0.028..0.029 rows=1 loops=3095201)
         Index Cond: ((pubid)::text = (auth.pubid)::text)
 Planning time: 0.647 ms
 Execution time: 98634.857 ms

Q2:
 Nested Loop  (cost=0.43..65701.99 rows=24 width=67) 
 (actual time=309.784..503.737 rows=183 loops=1)
   ->  Seq Scan on auth  (cost=0.00..65499.01 rows=24 width=23) 
       (actual time=309.720..498.274 rows=183 loops=1)
         Filter: ((name)::text = 'Divesh Srivastava'::text)
         Rows Removed by Filter: 3095018
   ->  Index Scan using pubidp on publ_i  (cost=0.43..8.45 rows=1 width=89) 
       (actual time=0.027..0.028 rows=1 loops=183)
         Index Cond: ((pubid)::text = (auth.pubid)::text)
 Planning time: 0.137 ms
 Execution time: 503.837 ms

\end{verbatim}
}

\noindent Index on {\tt Auth.pubID} (Q1/Q2):
{\small
\begin{verbatim}
Q1:
 Nested Loop  (cost=0.43..849985.42 rows=3095201 width=82) 
 (actual time=0.066..45483.489 rows=3095201 loops=1)
   ->  Seq Scan on publ  (cost=0.00..34694.14 rows=1233214 width=89) 
       (actual time=0.009..703.605 rows=1233214 loops=1)
   ->  Index Scan using pubida on auth_i  (cost=0.43..0.63 rows=3 width=38) 
       (actual time=0.026..0.033 rows=3 loops=1233214)
         Index Cond: ((pubid)::text = (publ.pubid)::text)
 Planning time: 0.432 ms
 Execution time: 46672.318 ms

Q2:
 Nested Loop  (cost=0.00..562648.46 rows=25 width=67) 
 (actual time=15405.220..193800.011 rows=183 loops=1)
   Join Filter: ((auth_i.pubid)::text = (publ.pubid)::text)
   Rows Removed by Join Filter: 225677979
   ->  Seq Scan on publ  (cost=0.00..34694.14 rows=1233214 width=89) 
       (actual time=0.012..689.233 rows=1233214 loops=1)
   ->  Materialize  (cost=0.00..65499.14 rows=25 width=23) 
       (actual time=0.000..0.071 rows=183 loops=1233214)
         ->  Seq Scan on auth_i  (cost=0.00..65499.01 rows=25 width=23) 
             (actual time=104.495..507.970 rows=183 loops=1)
               Filter: ((name)::text = 'Divesh Srivastava'::text)
               Rows Removed by Filter: 3095018
 Planning time: 0.135 ms
 Execution time: 193800.146 ms

\end{verbatim}
}

\noindent Index on {\tt Auth.pubID} and {\tt Auth.pubID} (Q1/Q2):
{\small
\begin{verbatim}
Q1:
 Nested Loop  (cost=0.43..850049.42 rows=3095201 width=82) 
 (actual time=0.101..55181.800 rows=3095201 loops=1)
   ->  Seq Scan on publ_i  (cost=0.00..34758.14 rows=1233214 width=89) 
       (actual time=0.037..721.909 rows=1233214 loops=1)
   ->  Index Scan using pubida on auth_i  (cost=0.43..0.63 rows=3 width=38) 
       (actual time=0.032..0.041 rows=3 loops=1233214)
         Index Cond: ((pubid)::text = (publ_i.pubid)::text)
 Planning time: 0.148 ms
 Execution time: 56394.801 ms

Q2:
 Nested Loop  (cost=0.43..65710.45 rows=25 width=67) 
 (actual time=103.244..511.228 rows=183 loops=1)
   ->  Seq Scan on auth_i  (cost=0.00..65499.01 rows=25 width=23) 
       (actual time=103.176..505.597 rows=183 loops=1)
         Filter: ((name)::text = 'Divesh Srivastava'::text)
         Rows Removed by Filter: 3095018
   ->  Index Scan using pubidp on publ_i  (cost=0.43..8.45 rows=1 width=89) 
       (actual time=0.028..0.029 rows=1 loops=183)
         Index Cond: ((pubid)::text = (auth_i.pubid)::text)
 Planning time: 0.181 ms
 Execution time: 511.343 ms

\end{verbatim}
}

\paragraph{Discussion}
Discuss here your observations. Are the response times expected? Why / why
not? 

\section{Sort-Merge Join}

\paragraph{Response times}

\begin{flushleft}
\begin{tabular}{l|r|r}
  Indexes & Response time Q1 [ms] & Response time Q2 [ms] \\
  \hline
  no index & too long & 27886 \\
  two non-clustering indexes & 37306 & 508 \\
  two clustering indexes & 18891 & 515  \\
\end{tabular}
\end{flushleft}

\paragraph{Query plans}\mbox{}\\ 

\noindent No index (Q1/Q2):
{\small
\begin{verbatim}
Q1:
Merge Join  (cost=846625.43..906956.29 rows=3095201 width=83)
   Merge Cond: ((publ.pubid)::text = (auth.pubid)::text)
   ->  Sort  (cost=285913.35..288996.38 rows=1233214 width=90)
         Sort Key: publ.pubid
         ->  Seq Scan on publ  (cost=0.00..34694.14 rows=1233214 width=90)
   ->  Materialize  (cost=560711.47..576187.47 rows=3095201 width=38)
         ->  Sort  (cost=560711.47..568449.47 rows=3095201 width=38)
               Sort Key: auth.pubid
               ->  Seq Scan on auth  (cost=0.00..57750.01 rows=3095201 width=38)

Q2:
Merge Join  (cost=351419.92..357590.96 rows=413 width=68) 
	    (actual time=24054.639..27857.060 rows=183 loops=1)
   Merge Cond: ((publ.pubid)::text = (auth.pubid)::text)
   ->  Sort  (cost=285913.35..288996.38 rows=1233214 width=90) 
             (actual time=23183.951..26169.586 rows=1229958 loops=1)
         Sort Key: publ.pubid
         Sort Method: external merge  Disk: 121400kB
         ->  Seq Scan on publ  (cost=0.00..34694.14 rows=1233214 width=90) 
                               (actual time=0.026..850.908 rows=1233214 loops=1)
   ->  Sort  (cost=65505.96..65506.99 rows=413 width=23) 
	     (actual time=519.464..519.550 rows=183 loops=1)
         Sort Key: auth.pubid
         Sort Method: quicksort  Memory: 39kB
         ->  Seq Scan on auth  (cost=0.00..65488.01 rows=413 width=23) 
                               (actual time=8.420..518.464 rows=183 loops=1)
               Filter: ((name)::text = 'Divesh Srivastava'::text)
               Rows Removed by Filter: 3095018
 Planning time: 0.351 ms
 Execution time: 27886.590 ms
\end{verbatim}
Naturally, if there are no indexes present on the merge-attributes, both relations have to be sorted on the merge-attribute. This takes place in the execution of both queries.
In Query 1 we see a sorting of both relations followed by a merge-join on the specified attribute (The Query took extremely long, so the execution was stopped).
query 2 is much faster since first a selection is consucted using a linear scan since there is no index on name. After that both relations are again sorted which is a lot faster
since the second only contains publications by one author. Just like in query 1 the system then uses a merge-join to get the result.
}

\noindent Two non-clustering indexes (Q1/Q2):
{\small
\begin{verbatim}
Q1:
Merge Join  (cost=0.86..263625.34 rows=3095201 width=83) 
            (actual time=0.038..36048.698 rows=3095201 loops=1)
   Merge Cond: ((publ.pubid)::text = (auth.pubid)::text)
   ->  Index Scan using pubidpubl on publ  (cost=0.43..73055.43 rows=1233214 width=90) 
					   (actual time=0.006..6608.030 rows=1233208 loops=1)
   ->  Index Scan using pubidauth on auth  (cost=0.43..148974.61 rows=3095201 width=38) 
				           (actual time=0.005..13439.998 rows=3095201 loops=1)
 Planning time: 0.768 ms
 Execution time: 37306.638 ms

Q2:
Merge Join  (cost=65499.99..141460.64 rows=24 width=68) 
	    (actual time=508.011..508.011 rows=0 loops=1)
   Merge Cond: ((publ.pubid)::text = (auth.pubid)::text)
   ->  Index Scan using pubidpubl on publ  (cost=0.43..73055.43 rows=1233214 width=90) 
					   (actual time=0.006..0.006 rows=1 loops=1)
   ->  Sort  (cost=65499.56..65499.62 rows=24 width=23) 
	     (actual time=508.002..508.002 rows=0 loops=1)
         Sort Key: auth.pubid
         Sort Method: quicksort  Memory: 25kB
         ->  Seq Scan on auth  (cost=0.00..65499.01 rows=24 width=23) 
			       (actual time=507.995..507.995 rows=0 loops=1)
               Filter: ((name)::text = 'Divesh Srivastav'::text)
               Rows Removed by Filter: 3095201
 Planning time: 0.528 ms
 Execution time: 508.045 ms
\end{verbatim}
In this case although we do not have physical sorting of the data itself, we can use the index (always sorted) to access the tables. This eliminates the need for sorting.
In query 1 we see that sorting has completely been removed. The system uses both non-clustering indexes to conduct a merge-join. This of course speeds up the query dramatically.
In query 2 we still see some sorting which is only due to the fact that the result eturned by the sequential scan (selection of name) can not be assumed to be sorted. However this sort
is only carried out on a small amount of tuples. afterwards the merge-join is executed using the index of the first elation and the sorted second relation.
This method is still not optimal since only the index is sorted whereas the actual tupes are still spread out across the disk which means that tuples with the similar attribute-values are not necessaily
close to each other. This prohibits the system from performing sequential reads which slows down query execution compared to a clustering index.
}

\noindent Two clustering indexes  (Q1/Q2):
{\small
\begin{verbatim}
Q1:
Merge Join  (cost=0.86..263629.21 rows=3095201 width=83) 
	    (actual time=0.022..17687.042 rows=3095201 loops=1)
   Merge Cond: ((publ.pubid)::text = (auth.pubid)::text)
   ->  Index Scan using pubidpubl on publ  (cost=0.43..73057.97 rows=1233214 width=90) 
					   (actual time=0.005..917.866 rows=1233208 loops=1)
   ->  Index Scan using pubidauth on auth  (cost=0.43..148975.94 rows=3095201 width=38) 
					   (actual time=0.004..2033.002 rows=3095201 loops=1)
 Planning time: 0.606 ms
 Execution time: 18891.404 ms

Q2:
Merge Join  (cost=65500.99..141464.18 rows=24 width=68) 
	    (actual time=515.935..515.935 rows=0 loops=1)
   Merge Cond: ((publ.pubid)::text = (auth.pubid)::text)
   ->  Index Scan using pubidpubl on publ  (cost=0.43..73057.97 rows=1233214 width=90) 
    				           (actual time=0.006..0.006 rows=1 loops=1)
   ->  Sort  (cost=65500.56..65500.62 rows=24 width=23) 
	     (actual time=515.926..515.926 rows=0 loops=1)
         Sort Key: auth.pubid
         Sort Method: quicksort  Memory: 25kB
         ->  Seq Scan on auth  (cost=0.00..65500.01 rows=24 width=23) 
                               (actual time=515.919..515.919 rows=0 loops=1)
               Filter: ((name)::text = 'Divesh Srivastav'::text)
               Rows Removed by Filter: 3095201
 Planning time: 0.535 ms
 Execution time: 515.970 ms
\end{verbatim}
The increase in execution-speed cannot be explained by looking at the query plans alone since the look exactly the same as the ones in the non-clustered example. However the situation is a little different.
Since there are clusteringindexes on both relations they are of course physically sorted on the disk. Execution of both queries is exactly the same as in the previous example.
The queries aresped up since the physical sorted tuples can now be sequentially read from the disk, which gives a pretty significant speed boost as can be notced by comparing the execution durations.
}

\paragraph{Discussion}
The observations as already discussed for each query plan were all as expected. Using a merge join when there are no indexes present is of course extremely slow due to both relations having to be physically sorted.
A non-clustering index speeds things up since no more sorting is required. Tuples can be accessed through the index however sequential disk-read is not possible.
Finally using two clustered indexes eliminates the need for sorting and sequential disk-reads can be used. Looking at the first query we see huge gains in execution time due to the factors previously explained.
The second query is also hugely sped up by the use of indexes vs no index. However there is not a large difference in execution time between using clustering and non-clustering indexes. Since this is not the case
in query 1 is has to be caused by the selection. After the selection the system can no longer use the index on the second relation and has to sort the result of the selection.
This part is the same in both cases (clustering/non-clustering). So it is likely that the sequential scan and following sort dominate the execution time and the differende between seuqential- and random-read of the first
relation does not have a big impact.

\section{Hash Join}

\paragraph{Response times}

\begin{flushleft}
\begin{tabular}{l|r|r}
  Indexes & Response time Q1 [ms] & Response time [ms] Q2 \\
  \hline
  no index & 9427 & 1668 \\
\end{tabular}
\end{flushleft}

\paragraph{Query plans}\mbox{}\\ 

\noindent No Index (Q1/Q2):
{\small
\begin{verbatim}
Q1:
 Hash Join  (cost=68174.32..250399.34 rows=3095201 width=82) 
 (actual time=1834.128..8248.005 rows=3095201 loops=1)
   Hash Cond: ((auth.pubid)::text = (publ.pubid)::text)
   ->  Seq Scan on auth  (cost=0.00..57761.01 rows=3095201 width=38)
       (actual time=0.011..1413.767 rows=3095201 loops=1)
   ->  Hash  (cost=34694.14..34694.14 rows=1233214 width=89) 
       (actual time=1833.830..1833.830 rows=1233214 loops=1)
         Buckets: 4096  Batches: 64  Memory Usage: 2344kB
         ->  Seq Scan on publ  (cost=0.00..34694.14 rows=1233214 width=89)
             (actual time=0.042..790.847 rows=1233214 loops=1)
 Planning time: 0.445 ms
 Execution time: 9426.895 ms

Q2:
 Hash Join  (cost=65499.31..140273.15 rows=24 width=67) 
 (actual time=607.564..1667.832 rows=183 loops=1)
   Hash Cond: ((publ.pubid)::text = (auth.pubid)::text)
   ->  Seq Scan on publ  (cost=0.00..34694.14 rows=1233214 width=89) 
       (actual time=0.045..592.658 rows=1233214 loops=1)
   ->  Hash  (cost=65499.01..65499.01 rows=24 width=23) 
       (actual time=516.867..516.867 rows=183 loops=1)
         Buckets: 1024  Batches: 1  Memory Usage: 11kB
         ->  Seq Scan on auth  (cost=0.00..65499.01 rows=24 width=23) 
             (actual time=321.786..516.733 rows=183 loops=1)
               Filter: ((name)::text = 'Divesh Srivastava'::text)
               Rows Removed by Filter: 3095018
 Planning time: 0.111 ms
 Execution time: 1667.943 ms

\end{verbatim}

}

\paragraph{Discussion}

What do you think about the response time of the hash index vs.\ the
response times of sort-merge and index nested loop join for each of
the queries? Explain.

\bigskip

\noindent Time in hours per person: {\bf XXX}

\bigskip

\begin{center}
  \begin{tabular}{c}
    \hline
    {\bf Important:} Reference your information sources!
    \\\hline
  \end{tabular}
\end{center}

\end{document}
