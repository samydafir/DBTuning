\documentclass[11pt]{scrartcl}

\usepackage{url,float}

\title{
  \textbf{\large Database Tuning -- Assignment 3}\\
  Index Tuning
}

\author{
 A2\\
\large Baumgartner Dominik, 0920177 \\
\large Dafir Thomas Samy, 1331483 \\
\large Sch\"orgnhofer Kevin, 1421082
}

\begin{document}

\maketitle

\medskip
9.4/static/sql-cluster.html
\noindent\textbf{Database system and version:} {\it PostgreSQL 9.6.0}

\subsection*{Question 1: Index Data Structures} Which index data structures (e.g., $B^+$-tree
index) are supported?

B-tree, Hash, GiST, SP-GiST and GIN
\footnote{PostgreSQL 9.6
  Documentation, Chapter 11 Indexes,
  \url{https://www.postgresql.org/docs/9.6/static/indexes-types.html}}


\subsection*{Question 2: Clustered Indexes} Discuss how the system
supports clustered indexes, in particular:

\paragraph{2a)} How do you create a clustered index on {\tt ssnum}?
Show the query.\footnote{Give the queries for creating a hash index
  \emph{and} a $B^+$-tree index if both of them are supported.}

\smallskip

To create a clustered index in PostgreSQL you create an index first and cluster this index afterwards with the CLUSTER command. By default PostgreSQL creates a $B^+$-tree index. If you want a hash index, you have to specify it:\footnote{PostgreSQL 9.6
  Documentation, Chapter 11 Indexes,
  \url{https://www.postgresql.org/docs/9.6/static/indexes-types.html}}
\footnote{PostgreSQL 9.6
  Documentation, SQL Cluster,
  \url{https://www.postgresql.org/docs/9.6/static/sql-cluster.html}}

{\small
\begin{verbatim}
--creates a $B^+$-tree index and clusters it
CREATE INDEX ssnumE ON Employee (ssnum);
CLUSTER Emlpoyee USING ssnumE;
\\
--creates a hash index and clusters it
CREATE INDEX ssnumE ON Employee USING HASH (ssnum);
CLUSTER Emlpoyee USING ssnumE;
\end{verbatim}
}

\paragraph{2b)} Are clustered indexes on non-key attributes supported, e.g.,
on {\tt name}?  Show the query.

Yes they are supported. In fact postgres supports clustering tables using any attribute used in an index.
According to Postgres documentation there is no limitation to which index-type can be used.\footnote{PostgreSQL 9.6
  Documentation, SQL Cluster,
  \url{https://www.postgresql.org/docs/9.6/static/sql-cluster.html}}

\begin{verbatim}
CLUSTER table USING index_name;
\end{verbatim}


\paragraph{2c)} Is the clustered index dense or sparse?

There is no answer to this question in the Postgres Documentation.
So we derived a solution which seems logical:
Every index which we create in Postgres is a non-clustered one, and hence dense\footnote{DB-Tuning Lecture Notes, Index Tuning, Page 17}. With the $CLUSTER$ command we just cluster the existing data to match the index, which only rearranges the data entries and pointers and does not change the type (dense/spares) of the index. So the conclusion is, that the clustered indexes in Postgres are dense.

\paragraph{2d)} How does the system deal with overflows in clustered indexes?
How is the fill factor controlled?

The system does not resolve overflows at all. If overflows occur the table has to be manually clustered. Order preservation
can be achieved by setting the fillfactor to a value below 100\%. This enables the insertion of matching new data into pages
inserted in the ordered structure. The table has to be reclustered.\footnote{PostgreSQL 9.6
  Documentation, SQL Cluster,
  \url{https://www.postgresql.org/docs/9.6/static/sql-cluster.html}}


An index with a specified fillfactor is created as follows:\footnote{PostgreSQL 9.6
  Documentation, SQL Create Index,
  \url{https://www.postgresql.org/docs/9.6/static/sql-createindex.html}}
\begin{verbatim}
CREATE UNIQUE INDEX index_title ON table (attribute) WITH (fillfactor = value);
\end{verbatim}



\paragraph{2e)} Discuss any further characteristics of the system
related to clustered indexes that are relevant to a database
tuner?

Tables using a clustered index have to be reclustered after every update to ensure they stay clustered.
If a cluster is in progress an ACCESS EXCLUSIVE lock is acquired meaning no other read or write operation can be executed
on the table.
Since clustering a table according to an index sorts the table physically on the index attribute there can only be one clustered index per table.
Furthermore the $CLUSTER$ instruction is not part of standard SQL but a specific feature of Postgresql.

Clustering can be done using either a sequential table-scan (plus sorting) or an index scan. Both methods require a certain amount of diskspace: tablesize + indexsize for an index scan, double tablesize + indexsize for sequential table-scan.
The method is selected automatically by the dbs whereas a sequential scan is often faster than an index scan. Due to the disk space requirements for a cluster it is advisable to set the $maintenance work mem$ to a reasonable size.\footnote{PostgreSQL 9.6
  Documentation, SQL Cluster,
  \url{https://www.postgresql.org/docs/9.6/static/sql-cluster.html}}

\subsection*{Question 3: Non-Clustered Indexes}

Discuss how the system supports non-clustered indexes, in
particular:

\paragraph{3a)} How do you create a non-clustered index on {\tt
  (dept,salary)}? Show the query.

In Postgresql all indexes are non-clustered. Clustering has to be done manually.
Postgres creates all indexes as B+ trees by default. If a hash index is desired it has to be specified in the query\footnote{PostgreSQL 9.6
  Documentation, SQL Create Index,
  \url{https://www.postgresql.org/docs/9.6/static/sql-createindex.html}}.

{\small
\begin{verbatim}
--creates a B+-tree index
CREATE INDEX deptSalary ON Employee (dept, salary);

--creates a hash index
CREATE INDEX deptSalary ON Employee USING HASH (dept, salary);
\end{verbatim}
}

\paragraph{3b)} Can the system take advantage of covering indexes? What if the
index covers the query, but the condition is not a prefix of the
attribute sequence {\tt (dept,salary)}?

The system takes advantage of covering indexes if the index covers the query and the condition is a prefix of the attribute sequence. If the condition is not a prefix of the attribute sequence the use of the index depends heavily on the type of query e.g. range queries hardly use the index.

\paragraph{3c)} Discuss any further characteristics of the system related to
non-clustered indexes that are relevant to a database tuner?

\begin{itemize}
 \item A non-clustered index is the standard index type in Postgres
 \item There can be more than one non-clustered index per table
 \item Faster inserts, because the position where the data gets inserted does not matter
 \item $INCLUDE$ command is available, which enables us to include non-key parameters on the leaf level (not supported in Postgres)
\end{itemize}

\subsection*{Question 4: Key Compression and Page Size} If your system
supports $B^+$-trees, what kind of key compression (if any) does it
support?  How large is the default disk page? Can it be changed?
\\
PostgreSQL does not have key compression in $B^+$-trees.\\
PostgreSQL uses a fixed page size (commonly 8 kB), and does not allow tuples to span multiple pages, according to its documentation.
A different page size can be selected when compiling the server.
\smallskip

\bigskip

\noindent Time in hours per person: {\bf 4}

\bigskip

\begin{center}
  \begin{tabular}{c}
    \hline
    {\bf Important:} Reference your information sources!
    \\\hline
  \end{tabular}
\end{center}

\end{document}
