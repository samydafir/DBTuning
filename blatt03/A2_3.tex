\documentclass[11pt]{scrartcl}

\title{
  \textbf{\large Database Tuning -- Assignment 3}\\
  Index Tuning
}

\author{
 A2\\
\large Baumgartner Dominik, 0920177 \\
\large Dafir Thomas Samy, 1331483 \\
\large Sch\"orgnhofer Kevin, 1421082
}

\begin{document}

\maketitle

\medskip
9.4/static/sql-cluster.html
\noindent\textbf{Database system and version:} {\it PostgreSQL 9.6.0}

\subsection*{Question 1: Index Data Structures} Which index data structures (e.g., $B^+$-tree
index) are supported?

\smallskip

B-tree, Hash, GiST, SP-GiST and GIN\\
https://www.postgresql.org/docs/9.4/static/indexes-types.html

\subsection*{Question 2: Clustered Indexes} Discuss how the system
supports clustered indexes, in particular:

\paragraph{2a)} How do you create a clustered index on {\tt ssnum}?
Show the query.\footnote{Give the queries for creating a hash index
  \emph{and} a $B^+$-tree index if both of them are supported.}

\smallskip

To create a clustered index in PostgreSQL you create an index first and cluster this index afterwards with the CLUSTER command. By default PostgreSQL creates a $B^+$-tree index. If you want a hash index, you have to specify it:

{\small
\begin{verbatim}
--creates a $B^+$-tree index and clusters it
CREATE INDEX ssnumE ON Employee (ssnum);
CLUSTER Emlpoyee USING ssnumE;
\\
--creates a hash index and clusters it
CREATE INDEX ssnumE ON Employee USING HASH (ssnum);
CLUSTER Emlpoyee USING ssnumE;
\end{verbatim}
}

https://www.postgresql.org/docs/9.6/static/indexes-types.html
https://www.postgresql.org/docs/9.6/static/sql-cluster.html

\paragraph{2b)} Are clustered indexes on non-key attributes supported, e.g.,
on {\tt name}?  Show the query.

\smallskip

Your answer\dots

{\small
\begin{verbatim}
SQL QUERY ...
\end{verbatim}
}

\paragraph{2c)} Is the clustered index dense or sparse?


\smallskip

Your answer\dots

\paragraph{2d)} How does the system deal with overflows in clustered indexes?
How is the fill factor controlled?

\smallskip

Your answer\dots

\paragraph{2e)} Discuss any further characteristics of the system
related to clustered indexes that are relevant to a database
tuner?

\smallskip

Your answer\dots

\subsection*{Question 3: Non-Clustered Indexes}

Discuss how the system supports non-clustered indexes, in
particular:

\paragraph{3a)} How do you create a non-clustered index on {\tt
  (dept,salary)}? Show the query.$^1$

\smallskip

Your answer\dots

{\small
\begin{verbatim}
SQL QUERY ...
\end{verbatim}
}

\paragraph{3b)} Can the system take advantage of covering indexes? What if the
index covers the query, but the condition is not a prefix of the
attribute sequence {\tt (dept,salary)}?


\smallskip

Your answer\dots

\paragraph{3c)} Discuss any further characteristics of the system related to
non-clustered indexes that are relevant to a database tuner?

\smallskip

Your answer\dots

\subsection*{Question 4: Key Compression and Page Size} If your system
supports $B^+$-trees, what kind of key compression (if any) does it
support?  How large is the default disk page? Can it be changed?


\smallskip

Your answer\dots


\bigskip

\noindent Time in hours per person: {\bf XXX}

\bigskip

\begin{center}
  \begin{tabular}{c}
    \hline
    {\bf Important:} Reference your information sources!
    \\\hline
  \end{tabular}
\end{center}

\end{document}
