\documentclass[11pt]{scrartcl}

\usepackage[top=2cm]{geometry}
\usepackage{url,hyperref}

\title{
  \textbf{\large Database Tuning -- Assignment 6}\\
  Concurrency Tuning
}

\author{
 Group Name (e.g. A1, B5, B3)\\
 \large Lastname1 Firstname1, StudentID1 \\
 \large Lastname2 Firstname2, StudentID2 \\
 \large Lastname3 Firstname3, StudentID3 
}

\begin{document}

\maketitle

\noindent
{\it Notes:}

\begin{itemize}
\item You will need to run transactions concurrently using threads in
  Java. See\\ \url{http://dbresearch.uni-salzburg.at/teaching/2016ws/dbt/account.zip}
  for an example.
\end{itemize}

\section*{Task 1}

\subsection*{Read Committed}

Throughput and correctness for solution (a) with serialization level
{\tt\small READ COMMITTED}.

\bigskip

\begin{tabular}{c|c|c}
  \#Concurrent Transactions & Throughput [transactions/sec] & Correctness
  \\\hline
  1 & & \\
  2 & & \\
  3 & & \\
  4 & & \\
  5 & & \\    
\end{tabular}

\medskip

\subsection*{Serializable}

Throughput and correctness for solution (a) with serialization level
{\tt\small SERIALIZABLE}.

\bigskip

\begin{tabular}{c|c|c}
  \#Concurrent Transactions & Throughput [transactions/sec] & Correctness
  \\\hline
  1 & & \\
  2 & & \\
  3 & & \\
  4 & & \\
  5 & & \\    
\end{tabular}

\medskip

\section*{Task 2}

\subsection*{Read Committed}

Throughput and correctness for solution (b) with serialization level
{\tt\small READ COMMITTED}.

\bigskip

\begin{tabular}{c|c|c}
  \#Concurrent Transactions & Throughput [transactions/sec] & Correctness
  \\\hline
  1 & & \\
  2 & & \\
  3 & & \\
  4 & & \\
  5 & & \\    
\end{tabular}

\medskip

\subsection*{Serializable}

Throughput and correctness for solution (b) with serialization level
{\tt\small SERIALIZABLE}.

\bigskip

\begin{tabular}{c|c|c}
  \#Concurrent Transactions & Throughput [transactions/sec] & Correctness
  \\\hline
  1 & & \\
  2 & & \\
  3 & & \\
  4 & & \\
  5 & & \\    
\end{tabular}

\medskip

\section*{Task 3: Discussion}

Discuss the outcome and explain the difference between the isolation
levels in PostgreSQL with respect to your experiment.

Explain {\bf with your own words} how PostgreSQL deals with updates in
the different isolation levels, within a transaction and within a
single SQL command. Explicitly explain why you got the experimental
results of Task~1 and Task~2.

\subsection*{Task 1}

Discuss outcome of task 1 here.

\subsection*{Task 2}

Discuss outcome of task 2 here.

\bigskip

\noindent Time in hours per person: {\bf XXX}

\bigskip

\begin{center}
  \begin{tabular}{c}
    \hline
    {\bf Important:} Reference your information sources!
    \\\hline
  \end{tabular}
\end{center}

\end{document}
