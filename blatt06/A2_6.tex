\documentclass[11pt]{scrartcl}

\usepackage[top=2cm]{geometry}
\usepackage{url,hyperref}

\title{
  \textbf{\large Database Tuning -- Assignment 6}\\
  Concurrency Tuning
}

\author{
A2\\
\large Baumgartner Dominik, 0920177 \\
\large Dafir Thomas Samy, 1331483 \\
\large Sch\"orgnhofer Kevin, 1421082 
}

\begin{document}

\maketitle

\section*{Setup}
All experiments were executed on the computers in the R\"UR. JDBC was used to send
transactions to the database server (biber). Time was measured using the $time$ shell command.

\section*{Task 1}

\subsection*{Read Committed}

Throughput and correctness for solution (a) with serialization level
{\tt\small READ COMMITTED}.

\bigskip

\begin{tabular}{c|c|c}
  \#Concurrent Transactions & Throughput [transactions/sec] & Correctness
  \\\hline
  1 & 18.9 & 100\% \\
  2 & 32.6 & 52\% \\
  3 & 45.4 & 70\% \\
  4 & 56.3 & 59\% \\
  5 & 64.1 & 55\% \\    
\end{tabular}

\medskip

\subsection*{Serializable}

Throughput and correctness for solution (a) with serialization level
{\tt\small SERIALIZABLE}.

\bigskip

\begin{tabular}{c|c|c}
  \#Concurrent Transactions & Throughput [transactions/sec] & Correctness
  \\\hline
   1 & 19.23 - 20.41 & 100 \% \\
  2 & 20.41 - 22.22 & 100 \% \\
  3 & 19.23 - 21.74 & 100 \%\\
  4 & 20.0 - 28.57& 100 \%\\\
  5 & 20.83 - 34.99 & 100 \% \\  
\end{tabular}

\medskip

\section*{Task 2}

\subsection*{Read Committed}

Throughput and correctness for solution (b) with serialization level
{\tt\small READ COMMITTED}.

\bigskip

\begin{tabular}{c|c|c}
  \#Concurrent Transactions & Throughput [transactions/sec] & Correctness
  \\\hline
  1 & 23.5 & 100\% \\
  2 & 39.1 & 100\% \\
  3 & 52.6 & 100\% \\
  4 & 58.9 & 100\% \\
  5 & 67.6 & 100\% \\    
\end{tabular}

\medskip

\subsection*{Serializable}

Throughput and correctness for solution (b) with serialization level
{\tt\small SERIALIZABLE}.

\bigskip

\begin{tabular}{c|c|c}
  \#Concurrent Transactions & Throughput [transactions/sec] & Correctness
  \\\hline
  1 & 21.74 - 23.8 & 100 \% \\
  2 & 26.31 - 40.0 & 100 \%\\
  3 & 27.78 - 40.0 & 100 \%\\
  4 & 35.71 - 45.45 & 100 \%\\
  5 & 37.04 - 43.47 & 100 \%\\    
\end{tabular}

\medskip

\section*{Task 3: Discussion}

Discuss the outcome and explain the difference between the isolation
levels in PostgreSQL with respect to your experiment.

Explain {\bf with your own words} how PostgreSQL deals with updates in
the different isolation levels, within a transaction and within a
single SQL command. Explicitly explain why you got the experimental
results of Task~1 and Task~2.

\subsection*{Task 1}

Discuss outcome of task 1 here.

\subsection*{Task 2}

For both isolation levels we get 100\% correctness, as expected.\\
For the isolation level serializable it is obvious that the concurrent transactions have no effects to each other (that is kind of the definition of this isolation level), which results in 100\% correctness.\\
To understand the 100\% correctness of the read committed isolation level, we have to take a look at the transaction: it consits of two update statements, which are both enclosed and do not relate on a value from a previous select statement (or similar).\\
The throughput increases as the number of concurrent transactions increases (as expected).\\
Compared to serializable the throughput of read committed is higher, because a concurrent transaction B which starts after transaction A, waits until A commits and commits then itself. With serializable, transaction B aborts after A commits. This leads to way more queries needed for the isolation level serializable, hence to a lower throughput.

\bigskip

\noindent Time in hours per person: {\bf XXX}

\bigskip

\begin{center}
  \begin{tabular}{c}
    \hline
    {\bf Important:} Reference your information sources!
    \\\hline
  \end{tabular}
\end{center}

\end{document}
