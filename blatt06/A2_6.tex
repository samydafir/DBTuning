\documentclass[11pt]{scrartcl}

\usepackage[top=2cm]{geometry}
\usepackage{url,hyperref}

\title{
  \textbf{\large Database Tuning -- Assignment 6}\\
  Concurrency Tuning
}

\author{
 Group Name (e.g. A1, B5, B3)\\
 \large Lastname1 Firstname1, StudentID1 \\
 \large Lastname2 Firstname2, StudentID2 \\
 \large Lastname3 Firstname3, StudentID3 
}

\begin{document}

\maketitle

\noindent
{\it Notes:}

\begin{itemize}
\item You will need to run transactions concurrently using threads in
  Java. See\\ \url{http://dbresearch.uni-salzburg.at/teaching/2016ws/dbt/account.zip}
  for an example.
\end{itemize}

\section*{Setup}
All experiments were executed on the computers in the R\"UR. JDBC was used to send
transactions to the database server (biber). Time was measured using the $time$ shell command.

\section*{Task 1}

\subsection*{Read Committed}

Throughput and correctness for solution (a) with serialization level
{\tt\small READ COMMITTED}.

\bigskip

\begin{tabular}{c|c|c}
  \#Concurrent Transactions & Throughput [transactions/sec] & Correctness
  \\\hline
  1 & 18.9 & 100\% \\
  2 & 32.6 & 52\% \\
  3 & 45.4 & 70\% \\
  4 & 56.3 & 59\% \\
  5 & 64.1 & 55\% \\    
\end{tabular}

\medskip

\subsection*{Serializable}

Throughput and correctness for solution (a) with serialization level
{\tt\small SERIALIZABLE}.

\bigskip

\begin{tabular}{c|c|c}
  \#Concurrent Transactions & Throughput [transactions/sec] & Correctness
  \\\hline
   1 & 19.23 - 20.41 & 100 \% \\
  2 & 20.41 - 22.22 & 100 \% \\
  3 & 19.23 - 21.74 & 100 \%\\
  4 & 20.0 - 28.57& 100 \%\\\
  5 & 20.83 - 34.99 & 100 \% \\  
\end{tabular}

\medskip

\section*{Task 2}

\subsection*{Read Committed}

Throughput and correctness for solution (b) with serialization level
{\tt\small READ COMMITTED}.

\bigskip

\begin{tabular}{c|c|c}
  \#Concurrent Transactions & Throughput [transactions/sec] & Correctness
  \\\hline
  1 & 23.5 & 100\% \\
  2 & 39.1 & 100\% \\
  3 & 52.6 & 100\% \\
  4 & 58.9 & 100\% \\
  5 & 67.6 & 100\% \\    
\end{tabular}

\medskip

\subsection*{Serializable}

Throughput and correctness for solution (b) with serialization level
{\tt\small SERIALIZABLE}.

\bigskip

\begin{tabular}{c|c|c}
  \#Concurrent Transactions & Throughput [transactions/sec] & Correctness
  \\\hline
  1 & 21,74 - 23.8 & 100 \% \\
  2 & 26.31 - 40.0 & 100 \%\\
  3 & 27.78 - 40.0 & 100 \%\\
  4 & 35.71 - 45.45 & 100 \%\\
  5 & 37.04 - 43.47 & 100 \%\\    
\end{tabular}

\medskip

\section*{Task 3: Discussion}

\subsection*{Task 1}
As expected using isolation level read committed led to a much higher throughput with increasing number of concurrent transactions. In the case of serializable the gain was not as dramatic. This is due to the conflicts caused by the serializable isolation level leading to transactions being executed several times before committing which leads to lower throughput.
Error rates are also as expected: Using serializable leads to no errors at all since it yields the same results as a serial execution of all transactions. In the case of read committed non repeatable reads occur since every transaction sees the database as it was at the time the transaction started. E.g. if a transaction manipulates $balance$ by subtracting 1 it does not notice that other transactions might have done the same in the meantime. When this transaction commits it writes the old value decreased by 1 in to the db. This leads to calculation errors but increases the throughput since no transactions are rolled back and re-executed.

Serializable uses strict 2-phase locking for read and write operations: only one transaction can read/write data at a time. Locks are released after a transaction commits. Transactions trying to modify the same dataset as the current transaction have to wait until it commits. Any changes done by other transactions are rolled back.

\subsection*{Task 2}

Discuss outcome of task 2 here.


\bigskip

\noindent Time in hours per person: {\bf XXX}

\bigskip

\begin{center}
  \begin{tabular}{c}
    \hline
    {\bf Important:} Reference your information sources!
    \\\hline
  \end{tabular}
\end{center}

\end{document}
