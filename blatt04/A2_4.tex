\documentclass[11pt]{scrartcl}

\usepackage[top=2cm]{geometry}
%\pagestyle{empty}

\title{
  \textbf{\large Database Tuning -- Assignment 4}\\
  Index Tuning
}

\author{
	A2\\
	\large Baumgartner Dominik, 0920177 \\
	\large Dafir Thomas Samy, 1331483 \\
	\large Sch\"orgnhofer Kevin, 1421082
}

\begin{document}

\maketitle

\noindent
{\it Notes:}
\begin{itemize}\itemsep=0pt
\item Do not forget to run {\tt ANALYZE tablename} after creating or
  changing a table.
\item Use {\tt EXPLAIN ANALYZE} for the query plans that you display in the report.
\end{itemize}


\section{Experimental Setup}

We used a small java program to send our queries to the database server. The java program was running on the computers of the R\"UR.\\
The time measurement starts before we start sending the queries of one type and one table (different tables for different indexes) and stops afterwards. We repeat this for each query type and for each table.

\section{Clustered B$^+$-Tree Index}

\paragraph{Point Query}

{\small
\begin{verbatim}
SELECT * FROM Publ WHERE pubID = ...
\end{verbatim}
}

\noindent
For this query we created a text file containing about 20000 randomly selected pubids from the publ.tsv file. Queries were then executed for each entry in that file.

\smallskip\noindent
Runtime: 65155ms. \\
Queries: 19728.
Throughput: about 302 queries/sec.

\smallskip\noindent
Query plan (for one of the queries):
{\small
\begin{verbatim}
Index Scan using pubidcb on publ_cb  (cost=0.43..8.45 rows=1 width=113) (actual time=0.061..0.061 rows=1 loops=1)
Index Cond: ((pubid)::text = 'books/idea/encyclopediaDB2005/Manjunath05'::text)
Planning time: 0.314 ms
Execution time: 0.079 ms
\end{verbatim}
An index scan is used to taverse the B$^+$-tree index and fint the tuple wth the specified value of pubId 
}

\paragraph{Multipoint Query -- Low Selectivity}

Repeat the following query multiple times with different conditions for {\tt booktitle}.

{\small
\begin{verbatim}
SELECT * FROM Publ WHERE booktitle = ...
\end{verbatim}
}

\noindent
Which conditions did you use?\\
We used a equality condition where the index matches a given string from the booktitle row of the publ.tsv file, like 'Software Engineering (Workshops)'.

\smallskip\noindent
Show the runtime results and compute the throughput.\\
For the Clustered B$^+$-Tree Index we achieved a runtime from 93660$ms$ to 101960$ms$ with 7190 searched values.
This leads to a throughput of 70,5$\frac{queries}{second}$ to 76,8$\frac{queries}{second}$.\\

Query plan (for one of the queries):
\begin{verbatim}
EXPLAIN SELECT * FROM publ_cb WHERE booktitle = 'Software Engineering (Workshops)';

 Index Scan using booktitlecb on publ_cb  (cost=0.43..14.60 rows=181 width=112)
   Index Cond: ((booktitle)::text = 'Software Engineering (Workshops)'::text)
(2 rows)
\end{verbatim}

In this execution plan an index scan is used. Due to the sorted attributes the cost are low.

\paragraph{Multipoint Query -- High Selectivity}

Repeat the following query multiple times with different conditions for {\tt year}.

{\small
\begin{verbatim}
SELECT * FROM Publ WHERE year = ...
\end{verbatim}
}

\noindent
Which conditions did you use?\\
The conditions were imported from a file which contains the years from 1920-2020 repeated up to a total amount of 1500 years.

\smallskip\noindent
Show the runtime results and compute the throughput.\\
Runtime (ms): 86316
Throughput (q/s): 17.37800639510635

\smallskip\noindent
Query plan (for one of the queries):
{\small
\begin{verbatim}
 Bitmap Heap Scan on publ_cb  (cost=45.98..6695.42 rows=2265 width=112) (actual time=1.081..6.685 rows=9182 loops=1)
   Recheck Cond: ((year)::text = '1986'::text)
   Heap Blocks: exact=160
   ->  Bitmap Index Scan on yearcb  (cost=0.00..45.41 rows=2265 width=0) (actual time=1.049..1.049 rows=9182 loops=1)
         Index Cond: ((year)::text = '1986'::text)
 Planning time: 0.484 ms
 Execution time: 9.974 ms
\end{verbatim}
}
\newpage
\section{Non-Clustered B$^+$-Tree Index}

\noindent \emph{Note:} Make sure the data is not physically ordered by
the indexed attributes due to the clustering index that you created
before.

\paragraph{Point Query}

{\small
\begin{verbatim}
SELECT * FROM Publ WHERE pubID = ...
\end{verbatim}
}

\noindent
For this query we created a text file containing about 20000 randomly selected pubids from the publ.tsv file. Queries were then executed for each entry in that file.

\smallskip\noindent
Runtime: 64573ms. \\
Queries: 19728.
Throughput: about 305 queries/sec.

\smallskip\noindent
Query plan (for one of the queries):
{\small
\begin{verbatim}
Index Scan using pubidb on publ_b  (cost=0.43..8.45 rows=1 width=112) (actual time=0.071..0.072 rows=1 loops=1)
Index Cond: ((pubid)::text = 'books/idea/encyclopediaDB2005/Manjunath05'::text)
Planning time: 0.405 ms
Execution time: 0.092 ms
\end{verbatim}
Just like in the case of the clustered B$^+$-tree index an index scan is used.
}


\paragraph{Multipoint Query -- Low Selectivity}

Repeat the following query multiple times with different conditions for {\tt booktitle}.

{\small
\begin{verbatim}
SELECT * FROM Publ WHERE booktitle = ...
\end{verbatim}
}

\noindent
Which conditions did you use?\\
We used a equality condition where the index matches a given string from the booktitle row of the publ.tsv file, like 'Software Engineering (Workshops)'.

\smallskip\noindent
Show the runtime results and compute the throughput.\\
For the Non-Clustered B$^+$-Tree Index we achieved a runtime from 99160$ms$ to 113470$ms$ with 7190 searched values.
This leads to a throughput of 63,4$\frac{queries}{second}$ to 72,5$\frac{queries}{second}$.\\

\smallskip\noindent
Query plan (for one of the queries):
\begin{verbatim}
EXPLAIN SELECT * FROM publ_b WHERE booktitle = 'Software Engineering (Workshops)';

 Index Scan using booktitleb on publ_b  (cost=0.43..577.16 rows=181 width=113)
   Index Cond: ((booktitle)::text = 'Software Engineering (Workshops)'::text)
(2 rows)
\end{verbatim}

This execution plan is almost the same as the one for the Clustered B$^+$-Tree. It also uses an index scan, but the costs are higher.



\paragraph{Multipoint Query -- High Selectivity}

Repeat the following query multiple times with different conditions for {\tt year}.

{\small
\begin{verbatim}
SELECT * FROM Publ WHERE year = ...
\end{verbatim}
}

\noindent
Which conditions did you use?\\
The conditions were imported from a file which contains the years from 1920-2020 repeated up to a total amount of 1500 years.

\smallskip\noindent
Show the runtime results and compute the throughput.\\
Runtime (ms): 83157
Throughput (q/s): 18.038168765106967

\smallskip\noindent
Query plan (for one of the queries):
{\small
\begin{verbatim}
 Bitmap Heap Scan on publ_b  (cost=55.01..7010.65 rows=2398 width=112) (actual time=2.148..34.510 rows=9182 loops=1)
   Recheck Cond: ((year)::text = '1986'::text)
   Heap Blocks: exact=3258
   ->  Bitmap Index Scan on yearb  (cost=0.00..54.41 rows=2398 width=0) (actual time=1.620..1.620 rows=9182 loops=1)
         Index Cond: ((year)::text = '1986'::text)
 Planning time: 0.301 ms
 Execution time: 38.380 ms
\end{verbatim}
}

\section{Non-Clustered Hash Index}

\noindent \emph{Note:} Make sure the data is not physically ordered by
the indexed attributes due to the clustering index that you created
before.

\paragraph{Point Query}

{\small
\begin{verbatim}
SELECT * FROM Publ WHERE pubID = ...
\end{verbatim}
}

\noindent
For this query we created a text file containing about 20000 randomly selected pubids from the publ.tsv file. Queries were then executed for each entry in that file.

\smallskip\noindent
Runtime: 64503ms. \\
Queries: 19728.
Throughput: about 305 queries/sec.

\smallskip\noindent
Query plan (for one of the queries):
{\small
\begin{verbatim}
Index Scan using pubidh on publ_h  (cost=0.00..8.02 rows=1 width=113) (actual time=0.037..0.038 rows=1 loops=1)
Index Cond: ((pubid)::text = 'books/idea/encyclopediaDB2005/Manjunath05'::text)
Planning time: 0.368 ms
Execution time: 0.060 ms
\end{verbatim}
In this case an index scan is used again which actually means that the hash value of the specified attribute is computed
and the corresponding bucket is searched for a matching tuple.
}


\paragraph{Multipoint Query -- Low Selectivity}

Repeat the following query multiple times with different conditions for {\tt booktitle}.

{\small
\begin{verbatim}
SELECT * FROM Publ WHERE booktitle = ...
\end{verbatim}
}

\noindent
Which conditions did you use?\\
We used a equality condition where the index matches a given string from the booktitle row of the publ.tsv file, like 'Software Engineering (Workshops)'.

\smallskip\noindent
Show the runtime results and compute the throughput.\\
For the Non-Clustered Hash Index we achieved a runtime from 124470$ms$ to 113470$ms$ with 7190 searched values.
This leads to a throughput of 57,8$\frac{queries}{second}$ to 76,9$\frac{queries}{second}$.\\

\smallskip\noindent
Query plan (for one of the queries):
\begin{verbatim}
EXPLAIN SELECT * FROM publ_h WHERE booktitle = 'Software Engineering (Workshops)';

 Bitmap Heap Scan on publ_h  (cost=5.37..668.34 rows=177 width=112)
   Recheck Cond: ((booktitle)::text = 'Software Engineering (Workshops)'::text)
   ->  Bitmap Index Scan on booktitleh  (cost=0.00..5.33 rows=177 width=0)
          Index Cond: ((booktitle)::text = 'Software Engineering (Workshops)'::text)
(4 rows)
\end{verbatim}
As seen in the plan, first the right bucket is retrieved. Then the booktitle is searched in the bucket.


\paragraph{Multipoint Query -- High Selectivity}

Repeat the following query multiple times with different conditions for {\tt year}.

{\small
\begin{verbatim}
SELECT * FROM Publ WHERE year = ...
\end{verbatim}
}

\noindent
Which conditions did you use?\\
The conditions were imported from a file which contains the years from 1920-2020 repeated up to a total amount of 1500 years.

\smallskip\noindent
Show the runtime results and compute the throughput.\\
Runtime (ms): 86386
Throughput (q/s): 17.363924710022456

\smallskip\noindent
Query plan (for one of the queries):
{\small
\begin{verbatim}
 Bitmap Heap Scan on publ_h  (cost=68.65..6436.08 rows=2149 width=113) (actual time=2.409..32.008 rows=9182 loops=1)
   Recheck Cond: ((year)::text = '1986'::text)
   Heap Blocks: exact=3258
   ->  Bitmap Index Scan on yearh  (cost=0.00..68.12 rows=2149 width=0) (actual time=1.824..1.824 rows=9182 loops=1)
         Index Cond: ((year)::text = '1986'::text)
 Planning time: 0.346 ms
 Execution time: 35.861 ms
\end{verbatim}
}


\section{Table Scan}

\noindent \emph{Note:} Make sure the data is not physically ordered by
the indexed attributes due to the clustering index that you created.
before.

\paragraph{Point Query}

{\small
\begin{verbatim}
SELECT * FROM Publ WHERE pubID = ...
\end{verbatim}
}

\noindent
For this query we created a text file containing about 300 randomly selected pubids from the publ.tsv file. Queries were then executed for each entry in that file. The reason we used a different dataset is the enormous amount of time 20000 sequential scan queries would have taken.

\smallskip\noindent
Runtime: 75518ms.\\
Queries: 297.\\
Throughput: about 3.9 queries/sec.

\smallskip\noindent
Query plan (for one of the queries):
{\small
\begin{verbatim}
Seq Scan on publ_s  (cost=0.00..37841.18 rows=1 width=113) (actual time=124.918..272.303 rows=1 loops=1)
Filter: ((pubid)::text = 'books/idea/encyclopediaDB2005/Manjunath05'::text)
Rows Removed by Filter: 1233213
Planning time: 0.029 ms
Execution time: 272.320 ms
\end{verbatim}
Since there is no index to be used, the optimizer creates a query plan using a sequential scan.
}


\paragraph{Multipoint Query -- Low Selectivity}

{\small
\begin{verbatim}
SELECT * FROM Publ WHERE booktitle = ...
\end{verbatim}
}

\noindent
Which conditions did you use?\\
We used a equality condition where the index matches a given string from the booktitle row of the publ.tsv file, like 'Software Engineering (Workshops)'.

\smallskip\noindent
Show the runtime results and compute the throughput.\\
For the Table Scan we achieved a runtime from 197998$ms$ to 201809$ms$ with 719 searched values.
This leads to a throughput of 3,6$\frac{queries}{second}$.\\

\smallskip\noindent
Query plan (for one of the queries):
\begin{verbatim}
EXPLAIN SELECT * FROM publ_s WHERE booktitle = 'Software Engineering (Workshops)';

 Seq Scan on publ_s  (cost=0.00..37843.18 rows=179 width=112)
   Filter: ((booktitle)::text = 'Software Engineering (Workshops)'::text)
(2 rows)
\end{verbatim}

As shown in this execution plan, it uses a sequal scan with booktitle on the table which ends in a high runtime.


\paragraph{Multipoint Query -- High Selectivity}

{\small
\begin{verbatim}
SELECT * FROM Publ WHERE year = ...
\end{verbatim}
}

\noindent
Which conditions did you use?\\
The conditions were imported from a file which contains the years from 1920-2020 repeated up to a total amount of 1500 years.

\smallskip\noindent
Show the runtime results and compute the throughput.\\
Runtime (ms): 485258
Throughput (q/s): 3.0911391465983042

\smallskip\noindent
Query plan (for one of the queries):
{\small
\begin{verbatim}
 Seq Scan on publ_s  (cost=0.00..37777.18 rows=2224 width=112) (actual time=0.041..354.600 rows=9182 loops=1)
   Filter: ((year)::text = '1986'::text)
   Rows Removed by Filter: 1224032
 Planning time: 0.746 ms
 Execution time: 358.390 ms
\end{verbatim}
}

\section{Discussion}

Give the throughput of the query types and index types in queries/second.
\begin{center}
  \begin{tabular}{c|c|c|c|c}
    & clustered & non-clust.\ B$^+$-tree & non-clust.\ hash & table scan \\
    \hline
    point ({\tt pubID}) & 302 & 305 & 305 & 3,9 \\
    \hline
    multipoint ({\tt booktitle}) & 73,14 & 69,37 & 69,73 & 3,6\\
    \hline
    multipoint  ({\tt year}) & 17,38 & 18,04 & 17,36 & 3,01 \\  
  \end{tabular}
\end{center}

\medskip

\subsection{Table Scan}
The results for the table scan are as expected for all query types. The database system can obviously not use an index an has to do a sequential scan over the whole table, which is slow, hence the throughput is low.
\subsection{Index Scans}
The results of the different index scans were nearly identical for each query type, which we kind of expected because of the low complexity of the queries.\\
The differences of the throughputs of different query types were also expected:\\
In the statistics of the DBS the variety of values for each attribute is stored. The single point query (pubID) returns only one tuple, whereas the multipoint queries have to find and return way more tuples (e.g. years about 12000 tuples), which is more expensive. This explains why the single point query has the highest throughput and the multipoint query containing year the lowest.\\
Our expectations were that the clustered B$^+$-tree would be faster than the non-clustered, which only occured on the first multipoint query (booktitle). This might be caused by the fact, that the non-clustered and the clustered B$^+$-tree both use a Bitmap Heap Scan which in our opinion makes no sense for the clustered one.\\
An explanation for this behaviour would be the amount of tuples returned by each query. In the booktitle-query only a small amount of tuples is returned thus an index scan is much less expensive than a bitmap scan. In the years-query a lot of tuples are returned which actually mean that there is a igh probability that blocks are accessed more than once which makes it far less expensive to conduct a bitmap scan using the index and scanning only marked blocks afterwards.\\ 
The differences between the throughput of the different index types of the single point query are in our opinion a little bit to small, because one tree traversel should be more expensive than one hash calculation. With about 20000 queries this should probably sum up to a higher difference. This is true for the other two query types as well. Especially for the (year) multipoint query it is weird that the DBS uses for each index a Bitmap Scan, hence does not take any advantage of the clustering.

\bigskip

\noindent Time in hours per person: {\bf XXX}

\bigskip

\begin{center}
  \begin{tabular}{c}
    \hline
    {\bf Important:} Reference your information sources!
    \\\hline
  \end{tabular}
\end{center}

\end{document}
