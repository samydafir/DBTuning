\documentclass[11pt]{scrartcl}

\usepackage{url,float}

\title{
  \textbf{\large Database Tuning -- Assignment 2}\\
  Query Tuning
}

\author{
 A2\\
\large Baumgartner Dominik, 0920177 \\
\large Dafir Thomas Samy, 1331483 \\
\large Sch\"orgnhofer Kevin, 1421082
}

\begin{document}

\maketitle

\subsection*{Experimental Data}

\paragraph{Creating Tables and Indexes}

SQL statements used to create the tables {\tt Employee}, {\tt
  Student}, and {\tt Techdept}, and the indexes on the tables:

{\small
\begin{verbatim}
CREATE TABLE Employee 
(
ssnum integer,
name character varying,
manager character varying,
dept character varying,
salary numeric(8,2),
numfriends smallint,
CONSTRAINT ssnumNameE PRIMARY KEY (ssnum, name)
);
CREATE UNIQUE INDEX ssnumE ON Employee (ssnum);
CREATE UNIQUE INDEX nameE ON Employee (name);
CREATE INDEX deptE ON Employee (dept);

CREATE TABLE Student 
(
ssnum integer,
name character varying,
course character varying,
grade smallint,
CONSTRAINT ssnumNameS PRIMARY KEY (ssnum, name)
);
CREATE UNIQUE INDEX ssnumS ON Student (ssnum);
CREATE UNIQUE INDEX nameS ON Student (name);

CREATE TABLE Techdept
(
dept character varying,
manager character varying,
location character varying,
CONSTRAINT deptT PRIMARY KEY (dept)
);
-- no explicit creation of a unique index necessary, because the primary key
creates implicitly an unique index on dept (according to postgre doc)
\end{verbatim}
}

\paragraph{Populating the Tables}

How did you fill the tables? What values did you use? Give a short
description of your program.

\subsection*{Query 1}

\paragraph{Original Query}

{\small
\begin{verbatim}
SELECT count(ssnum)
FROM Employee e1
WHERE salary > (SELECT AVG(e2.salary)
FROM Employee e2, Techdept
WHERE e2.dept = e1.dept
AND e2.dept = Techdept.dept);
\end{verbatim}
}

\paragraph{Rewritten Query}

{\small
\begin{verbatim}
SELECT count(ssnum)
FROM Techdept t, Employee e1, 
(SELECT dept, AVG(salary) as avgsalary FROM Employee GROUP BY dept) as e2
WHERE e1.dept = t.dept AND e1.salary > e2.avgsalary AND e1.dept = e2.dept;
\end{verbatim}
}

\paragraph{Evaluation of the Execution Plans}

Execution plan original query:

\begin{verbatim}
Aggregate  (cost=117285824.33..117285824.34 rows=1 width=4)
   ->  Seq Scan on employee e1  (cost=0.00..117285741.00 rows=33333 width=4)
         Filter: (salary > (SubPlan 1))
         SubPlan 1
           ->  Aggregate  (cost=1172.83..1172.84 rows=1 width=5)
                 ->  Nested Loop  (cost=117.23..1159.67 rows=5263 width=5)
                       ->  Index Only Scan using deptt on techdept  (cost=0.15..8.17 rows=1 width=32)
                             Index Cond: (dept = (e1.dept)::text)
                       ->  Bitmap Heap Scan on employee e2  (cost=117.08..1098.87 rows=5263 width=8)
                             Recheck Cond: ((dept)::text = (e1.dept)::text)
                             ->  Bitmap Index Scan on depte  (cost=0.00..115.77 rows=5263 width=0)
                                   Index Cond: ((dept)::text = (e1.dept)::text)
\end{verbatim}

Give an interpretation of the execution plan, i.e., describe how the
original query is evaluated.

Execution plan rewritten query:

\begin{verbatim}
Aggregate  (cost=4847.34..4847.35 rows=1 width=4)
   ->  Hash Join  (cost=2435.89..4764.01 rows=33333 width=4)
         Hash Cond: ((e1.dept)::text = (t.dept)::text)
         Join Filter: (e1.salary > e2.avgsalary)
         ->  Seq Scan on employee e1  (cost=0.00..1916.00 rows=100000 width=12)
         ->  Hash  (cost=2435.66..2435.66 rows=19 width=67)
               ->  Hash Join  (cost=2416.67..2435.66 rows=19 width=67)
                     Hash Cond: ((t.dept)::text = (e2.dept)::text)
                     ->  Seq Scan on techdept t  (cost=0.00..16.40 rows=640 width=32)
                     ->  Hash  (cost=2416.43..2416.43 rows=19 width=35)
                           ->  Subquery Scan on e2  (cost=2416.00..2416.43 rows=19 width=35)
                                 ->  HashAggregate  (cost=2416.00..2416.24 rows=19 width=8)
                                       Group Key: employee.dept
                                       ->  Seq Scan on employee  (cost=0.00..1916.00 rows=100000 width=8)
\end{verbatim}

Give an interpretation of the execution plan, i.e., describe how the
rewritten query is evaluated.

Discuss, how the execution plan changed between the original and the
rewritten query. In both the interpretation of the query plans and the
discussion focus on the crucial parts, i.e., the parts of the query
plans that cause major runtime differences.

\paragraph{Runtime} Discuss, why the rewritten query is (or is not)
faster than the original query.


\begin{table}[H]
  \begin{tabular}{l|r}
    & Runtime [ms] \\
   \hline
    Original query & 181654 \\
    Rewritten query & 95 \\
  \end{tabular}
\end{table}

\subsection*{Query 2}

\paragraph{Original Query}

{\small
\begin{verbatim}
SELECT DISTINCT ssnum FROM employee;
\end{verbatim}
}

\paragraph{Rewritten Query}

{\small
\begin{verbatim}
 SELECT ssnum FROM employee;
\end{verbatim}
}

\paragraph{Evaluation of the Execution Plans}

Execution plan original query:

\begin{verbatim}
Unique  (cost=0.29..3799.50 rows=100983 width=4)
   ->  Index Only Scan using ssnume on employee  (cost=0.29..3547.04 rows=100983
\end{verbatim}

Give an interpretation of the execution plan, i.e., describe how the
original query is evaluated.

Execution plan rewritten query:

\begin{verbatim}
Seq Scan on employee  (cost=0.00..1934.83 rows=100983 width=4)
\end{verbatim}

Give an interpretation of the execution plan, i.e., describe how the
rewritten query is evaluated.

Discuss, how the execution plan changed between the original and the
rewritten query. In both the interpretation of the query plans and the
discussion focus on the crucial parts, i.e., the parts of the query
plans that cause major runtime differences.

\paragraph{Runtime} Discuss, why the rewritten query is (or is not)
faster than the original query.


\begin{table}[H]
  \begin{tabular}{l|r}
    & Runtime [ms] \\
   \hline
    Original query & 103 \\
    Rewritten query & 72 \\
  \end{tabular}
\end{table}


  Time in hours per person: {\bf XXX}

\end{document}
